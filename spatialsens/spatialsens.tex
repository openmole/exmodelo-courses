\documentclass{../beamer_template/myBeamer}

\input{../beamer_template/header.tex}

\usepackage{pifont}
%\newcommand{\cmark}{\textcolor{green}\ding{51}}
%\newcommand{\xmark}{\textcolor{red}\ding{55}}
\newcommand{\cmark}{\ding{51}}
\newcommand{\xmark}{\ding{55}}

\title[Spatial sensitivity]{Spatial sensitivity analysis}
\subtitle{Course and practical application}
%\author[Short Author]{Author}
\date{September 2nd, 2020}
\institute{
\includegraphics[scale=2]{../beamer_template/figures/logos/openmole.png}}

\begin{document}



\begin{frame}[plain]
	\titlepage
\end{frame}
\addtocounter{framenumber}{-1}

\AtBeginSection[]
{
	\frame{
		\tableofcontents[currentsection, hideallsubsections]
	}
	\addtocounter{framenumber}{-1}
}


\sframe{Contents of this module}{

 %- Generation of spatial synthetic data
 %    - microscopic scale (buildings)
 %    - mesoscopic scale (population grid)
 %    - macroscopic scale (system of cities)
 %- Perturbation of real datasets
 %- Spatial indicators for model outputs
 %- Practice

	\tableofcontents

}




\section{Scientific context}

% - slides of evolutive urban theory
% - genesis of openmole etc.
% - recall simpoplocal / marius
% - deepening into openmole's roots ?

% + perspectives ? -> in opening ? ~ because practice




\sframe{Modeling and simulation in geography}{

% Cette présentation propose d'illustrer la question de la validation des modèles de simulation, dans le cas particulier de la géographie

%The role of simulation models in the production of knowledge has significantly shifted in recent years, accompanied with a transformation of practices, including methods and tools. This presentation aims at describing these mutations from the point of view of theoretical and quantitative geography.

\begin{columns}

	\begin{column}{0.5\textwidth}
	\centering
	
	\includegraphics[width=0.6\textwidth]{figures/openshaw.png}
	
	\footnotesize
\textit{Geographical analysis machine \cite{openshaw1987mark}}
	
	\medskip
	\hrule
	\medskip
	
	\includegraphics[width=0.55\textwidth,height=0.3\textheight]{figures/intro_RBD_lattice.png}
	
	\footnotesize
\textit{Hybrid urban morphogenesis \cite{raimbault2014hybrid}}
	

	\end{column}
	\vrule{}
	\begin{column}{0.5\textwidth}
	\centering
	
	\includegraphics[width=0.7\textwidth]{figures/simpop1.png}
	
	\footnotesize
\textit{Simpop 1 model\cite{sanders1997simpop}}

	\medskip

	\hrule
	
	\medskip

	\includegraphics[width=0.6\textwidth]{figures/setup_synth_1_tick100.png}
	
	\footnotesize
	\textit{SimpopNet model \cite{schmitt2014modelisation}}
	
	\end{column}


\end{columns}

}



\sframe{Geographical systems and complexity}{

% pour laquelle la prépondérance de l'espace augmente la complexité des modèles

\textit{Necessity of simulation models in geography induced by complexities of these systems ?}

\bigskip

\begin{itemize}
	\item Ontological complexity \cite{pumain2003approche}
	\item Dynamical complexity: non-ergodicity and path-dependency \cite{pumain2012urban}
	\item Complexity and co-evolution \cite{raimbault2019modeling}
	\item Complexity and emergence \cite{bedau2002downward}
\end{itemize}



}


\sframe{Theories and models}{

Historical succession of epistemologies in the case of systems of cities
\cite{varenne2017theories}: 

\medskip

\begin{enumerate}
	\item Deduction from theory (top-down): Christaller
	\item Induction from the empirical (bottom-up): Berry
	\item Towards an abductive epistemology (iterative interaction theoretical-empirical): Pumain
\end{enumerate}

\medskip

$\rightarrow$ simulation allows synthesis

}





\sframe{Towards new practices: ERC Geodivercity}{

% presentation generale de Geodivercity

\begin{columns}
\begin{column}{0.4\textwidth}
	\centering
	\includegraphics[width=\textwidth]{figures/urban-dynamics-simulation-models-geodivercity.png}
\end{column}
\begin{column}{0.6\textwidth}
	Development of evolutive urban theory \cite{pumain2018evolutionary}
	
	\medskip

	$\rightarrow$ Recurrent stylized facts on main systems of cities
	
	$\rightarrow$ Construction of simulation models (with an explicative purpose)
	
	$\rightarrow$ Tools and methods to explore simulation models
	
	\smallskip
	
	\includegraphics[width=\textwidth]{figures/openmole.png}
		
	
\end{column}
\end{columns}


}

\sframe{Evolutionary Urban Theory}{
\begin{columns}
\column{0.7\textwidth}
\centering
\includegraphics[height=0.9\textheight]{figures/evoltheory_scales}

\column{0.05\textwidth}

\hspace{0.5cm}

\column{0.25\textwidth}

\footnotesize

\textit{Systems of cities as co-evolutive systems in which interactions are crucial}
	
	\cite{pumain1997pour}
	
	\cite{pumain2008socio}
	
	\cite{pumain2018evolutionary}
\end{columns}
}


\sframe{Construction of Knowledge across Domains}{

% Existence of Knowledge Domains ?

% Taking a step back, emerges a typology of domains in which knowledge was created but also necessary for the other domains in the genesis of the Evolutive Urban Theory. The collection of data and construction of datasets is a first requirement for any further knowledge. From data are extracted empirical stylized facts, from which are induced theoretical hypotheses. Theory can then be tested for falsification, in the empirical domain but also through models, for example by doing targeted experiments in models of simulation. New methods are developed to better explore them. Tools are crucial at each step, to implement model, do data mining for example or collect and format data for example. The previous analysis reveals how these domains are interdependent, are in a sense \emph{co-evolutive}.

% illustration OpenMole / CalibProfile / Marius

\centering

\includegraphics[height=0.9\textheight]{figures/openmoleslide}


\nocite{baffi:tel-01389347}
\nocite{pumain2008socio}
\nocite{reuillon2013openmole}
\nocite{cottineau2015modular}
\nocite{swerts2017database}
\nocite{pumain1997pour}
\nocite{pumain2010theorie}

}

\sframe{Evolutive urban theory}{

% A particular entry is taken by the Evolutive Urban Theory [1] which postulates interactions between cities as main drivers of their growth.



\includegraphics[height=0.7\textheight]{figures/evolurbantheory.pdf}

\footnotesize
\textit{\cite{raimbault2017applied} Citation network analysis of key publications in the evolutive urban theory}

}



\sframe{Calibration of SimpopLocal}{

%Reverse approach: From output values to parameter values.

SimpopLocal model calibrated with distributed NSGA2 on grid \cite{schmitt2014half}

\medskip

%Formalising the expectations
\textit{Formalising the expectations as indicators: }

\centering
	
	\includegraphics[width=\textwidth]{figures/slocal_fitness.png}

}

\sframe{SimpopLocal and calibration profile}{

\textit{Computes the best calibration at fixed steps along one dimension.}
 \cite{reuillon2015}
 
 \medskip
 
 \includegraphics[width=0.48\textwidth]{figures/PdiffusionZ.png}
 \includegraphics[width=0.48\textwidth]{figures/InnovationLife.png}

}


\sframe{Unicity of mechanisms}{

%Automate the confrontation of alternative hypothesis / mechanisms.

\textit{Automate the confrontation of alternative hypothesis / mechanisms} \cite{cottineau2014evolution}

\medskip

\centering
	
	\includegraphics[width=\textwidth]{figures/simfamily.png}

}

\sframe{Multi-modeling (64 models)}{

%64 models

\centering

\includegraphics[width=0.8\textwidth]{figures/marius_complexification.png}

}


\sframe{Exemple of concurrent hypothesis}{


%Exchange mechanism:

%Market based
%Centrally planned

\textbf{Exchange mechanism: } market vs centralized

\medskip

%City growth mechanism:
%Purely inter-urban interactions
%Influenced by environmental situation

\textbf{City growth: } interurban interactions vs environmental situation

\medskip

\centering

\includegraphics[width=0.5\textwidth]{figures/system_mec.png}

}


\sframe{Calibration of model family}{

%
\textit{Compute the best set of parameters for all 64 models.}

\bigskip

\centering

\includegraphics[width=\textwidth]{figures/varius_big.png}

% http://shiny.parisgeo.cnrs.fr/VARIUS/

}




\sframe{Other example of multi-modeling}{

\textit{Benchmark of growth models for systems of cities} \cite{Raimbault_2020} %\cite{raimbault2018systematic}

\centering

\includegraphics[width=0.85\textwidth]{figures/CN_zoomed.png}

}






\sframe{PSE algorithm on the MARIUS model}{

\textit{Diversity of urban systems dynamics produced by the MARIUS model} \cite{10.1371/journal.pone.0138212}

\centering

\includegraphics[width=0.6\textwidth]{figures/pse_marius.png}

}






\sframe{Sensitivity analysis issues in geography}{

\textit{Classical problems in geography and spatial sciences :}

\medskip

\begin{itemize}
	\item Modifiable Areal Unit Problem
	\item Dependancy of processes to scale
	\item Spatial non-stationarity
	\item Fuzzy and noisy data
	\item Genericity/particularity 
\end{itemize}

}


\sframe{The Modifiable Areal Unit Problem}{

\begin{center}
	\includegraphics[height=0.75\textheight]{figures/maup.png}	
\end{center}


\medskip

\footnotesize

Wong, D. (2009). The modifiable areal unit problem (MAUP). The SAGE handbook of spatial analysis, 105, 23.

}


\sframe{Multiscalar systems}{

\textit{Processes specific to scales, coupling implies dedicated ontologies} \cite{raimbault2019towards}

\bigskip

\begin{center}
	\includegraphics[width=0.9\textwidth]{figures/multiscale_morph.pdf}
\end{center}




}

\sframe{Spatial non-stationarity}{

\textit{Spatial non-stationarity of road network properties} \cite{raimbault2019urban}

\bigskip

\centering

\includegraphics[width=\textwidth]{figures/nwindics_nonstat.png}

}


\sframe{Noisy spatial data}{

\textit{Assessment of data quality in OpenStreetMap} \cite{fan2014quality}

\medskip

\centering

\includegraphics[height=0.8\textheight]{figures/osm_data_accuracy.png}

}


\sframe{Genericity and specificity}{

\textit{Urban systems are simultaneously universal and particular} \cite{pumain2015multilevel}

\medskip

\centering

\includegraphics[height=0.8\textheight]{figures/multilevelpumain_zipf.png}

}



\sframe{Approach proposed by OpenMOLE}{

$\implies$ \textbf{spatial configuration are parameters too}

\medskip

\begin{itemize}
	\item ``\textit{Space matters}'': impact of spatial configuration on model behavior
	\item Model behaviors which are robust to spatial configuration
	\item Model behaviors which are robust to noise in parametrization real datasets
\end{itemize}

% - Space matters
% - Synthetic generators
% - Sensitivity to data noise
 
}









\section{Spatial synthetic data}



\sframe{General context}{

\justify

$\rightarrow$ coupling models with spatial configuration generators (spatial synthetic data) gives model sensitivity to space through sensitivity analysis of the coupled model

\bigskip

$\rightarrow$ synthetic urban forms resembling real configurations

\bigskip

$\rightarrow$ at different scales: microscopic (buildings), mesoscopic (population distribution), macroscopic (system of cities)

}



\sframe{Generating building layouts}{

\textit{At the microscopic scale (district): generating building layouts}

\nocite{raimbault2019generating}

\medskip

Raimbault, J., \& Perret, J. (2019). Generating urban morphologies at large scales. In Artificial Life Conference Proceedings (pp. 179-186). Cambridge: MIT Press.

\medskip

\begin{itemize}
	\item systematic comparison of simple processual generators
	\item introduction of morphological indicators
	\item calibration on sampled layouts from OpenStreetMap
\end{itemize}

}


\sframe{Quantifying urban form}{

% Simple descriptive indicators considered are (i) the total building density $A = \frac{1}{N}\cdot \sum_i s_i$; (ii) the number of buildings given by the number of connected components of $B$; (iii) the average building area, i.e. the average size of $B$ connected components; (iv) Moran index capturing spatial autocorrelation (see \cite{raimbault2018calibration} for its definition in a similar setting), with a simple inverse distance weight function; (v) average distance between non-empty points (which also captures a level of concentration).
% We also use indicators computed with the underlying networks: the average detour computed in the free space network $\bar{B}$, computed by randomly sampling 50 pairs of points in a connected component of $\bar{B}$ and computing the ratio between the network distance and the euclidian distance $d_{\bar{B}}/d_E$. This measures captures in a way the sinuosity of streets from a mobility viewpoint. We also consider the average size of open connected areas as the average size of the connected components of $\bar{B}$.
% steps for dilation + steps for erosion


Urban form indicators for building layouts:

\medskip

\begin{itemize}
	\item density, number of buildings, average area
	\item Moran index and average distance on rasterized representation
	\item average detour in the free space
	\item mathematical morphology indicators (steps for erosion and dilation) \cite{serra1994morphological}
\end{itemize}


}


\sframe{Generators}{

\textit{Complementary generators}

\medskip

\begin{center}
	\includegraphics[height=0.85\textheight]{figures/spatialsens_examplegenerators.png}
\end{center}


}

\sframe{Real configurations}{

\textit{Sampled districts from OpenStreetMap}

%\medskip

\begin{center}
	\includegraphics[height=0.85\textheight]{figures/spatialsens_exampleosm.png}
\end{center}

}


\sframe{Classification of urban forms}{

\begin{center}
   \includegraphics[height=0.9\textheight]{figures/spatialsens_osm_classification.png}
\end{center}

}




\sframe{Calibrated forms}{

\centering

\includegraphics[width=\textwidth]{figures/spatialsens_calib.png}

}


\sframe{Point cloud}{

\begin{center}
\includegraphics[height=0.9\textheight]{figures/spatialsens_points.png}
\end{center}

}


\sframe{Calibration}{

\includegraphics[width=\textwidth]{figures/spatialsens_distanceclasses.png}

\medskip

\textit{Why not use calibration heuristics? Open question of fitting a point cloud; issue of projecting in a reduced dimension space}

}


\sframe{Implementation in OpenMOLE}{

% spatialsmapling syntax 

\begin{center}
\includegraphics[width=\linewidth]{figures/spatialsampling.png}
\end{center}

\medskip

\justify

$\rightarrow$ other generators have their own primitives\\(\texttt{BlocksGridSpatialSampling},\\ \texttt{PercolationGridSpatialSampling}) and arguments (\textbf{see the documentation})

\medskip

$\rightarrow$ explore the script \texttt{1{\_}spatial{\_}sampling.oms}


}

\sframe{Practice: exploration of generators}{

Script \texttt{1{\_}spatial{\_}sampling.oms}

\medskip

\begin{enumerate}
	\item Use the \texttt{MatrixHook} and an \texttt{EmptyTask}	to visualise different types of grid
	\item Explore the generators applied to the ZOMBIE model at \url{https://om.exmodelo.org/spatialsens/}
	%\item Check the flatten spatial configurations (using the \texttt{flattenGrid} ScalaTask)
	\item Next step: Modify the script to plug random worlds into the model (later)
\end{enumerate}


}





\sframe{Synthetic population grids}{

\textit{At the mesoscopic scale: population grids}


\medskip


\begin{itemize}
	\item a reaction-diffusion model for population distributions
	\item urban form measures at the mesoscopic scale
	\item towards synthetic generators coupling population and road networks
\end{itemize}


\bigskip
\bigskip

\footnotesize

Raimbault, J. (2018). Calibration of a density-based model of urban morphogenesis. PloS one, 13(9), e0203516.

\nocite{raimbault2018calibration}

\smallskip

Raimbault, J. (2019). An urban morphogenesis model capturing interactions between networks and territories. In The Mathematics of Urban Morphology (pp. 383-409). Birkhäuser, Cham.

\nocite{raimbault2019urban}


}

\sframe{A simple Reaction-diffusion model}{


\justify

$\rightarrow$ Crucial role of the interplay between concentration forces and dispersion forces~\cite{fujita1996economics} in keeping Urban Systems at the border of chaos

\bigskip

$\rightarrow$ Potentiality of aggregation mechanisms (such as Simon model) to produce power laws %\cite{2016arXiv160806313S}

\bigskip

$\rightarrow$ Link with Reaction-diffusion approaches in Morphogenesis

\cite{turing1952chemical}

\bigskip

$\rightarrow$ Extension of a DLA-type model introduced by \cite{batty1991generating}, with simple abstract processes of population aggregation and diffusion

}


\sframe{Model Formalization}{

$\rightarrow$ Grid world with cell populations $(P_i(t))_{1\leq i\leq N^2}$.

\bigskip

$\rightarrow$ At each time step:

\begin{enumerate}
\item Population growth with exogenous rate $N_G$, attributed independently to a cell following a preferential attachment of strength $\alpha$
%\begin{equation}
%\Pb{P_i(t+1)=P_i(t)+1|P(t+1)=P(t)+1}=\frac{(P_i(t)/P(t))^{\alpha}}{\sum(P_i(t)/P(t))^{\alpha}}
%\end{equation}
%The attribution being uniformly drawn if all population are equal to 0.
\item Population is diffused $n_d$ times with strength $\beta$
\end{enumerate}

\bigskip

$\rightarrow$ Stopping criterion: fixed maximal population $P_m$.

%To avoid bord effects such as reflecting diffusion waves, border cells diffuse their due proportion outside of the world, implying that the total population at time $t$ is strictly smaller than $N_G\cdot t$.

\bigskip

$\rightarrow$ Output measured by morphological indicators: Moran index, average distance, rank-size hierarchy, entropy.


}


\sframe{Generating Population Distributions}{

\centering

\includegraphics[height=0.8\textheight]{figures/density_Fig2}

\footnotesize\textit{Examples of generated territorial shapes}

}


\sframe{Morphological indicators}{

\begin{enumerate}
\item Rank-size slope $\gamma$, given by $\ln \left( P_{\tilde{i}}/P_0\right) \sim k + \gamma\cdot \ln \left(\tilde{i}/i_0\right)$ where $\tilde{i}$ are the indexes of the distribution sorted in decreasing order.
\item Entropy of the distribution:
\begin{equation}
\mathcal{E} = \sum_{i=1}^{M}\frac{P_i}{P}\cdot \ln{\frac{P_i}{P}}
\end{equation}
$\mathcal{E}=0$ means that all the population is in one cell whereas $\mathcal{E}=0$ means that the population is uniformly distributed.
\item Spatial-autocorrelation given by Moran index, with simple spatial weights given by $w_{ij} = 1/d_{ij}$
\[
I = M \cdot \frac{\sum_{i\neq j} w_{ij} \left(P_i - \bar{P}\right)\cdot\left(P_j - \bar{P}\right)}{\sum_{i\neq j} w_{ij} \sum_{i}{\left( P_i - \bar{P}\right)}^2}
\]
\item Mean distance between individuals
\[
\bar{d} = \frac{1}{d_M}\cdot \sum_{i<j} \frac{P_i P_j}{P^2} \cdot d_{ij}
\]
where $d_M$ is a normalisation constant
\end{enumerate}



}



\sframe{Model behavior}{


\centering

\includegraphics[width=0.9\textwidth]{figures/density_Fig3}

\footnotesize\textit{Phase transitions of indicators unveiled by exploration of the parameter space (80000 parameter points, 10 repetitions each)}


}




\sframe{Path-dependency and frozen accidents}{

\centering

\includegraphics[width=0.8\textwidth]{figures/density_Fig4}

\footnotesize\textit{Illustration of path-dependence in a simplified one-dimensional version of the model: cell trajectories in time for 9 independent repetitions from the same initial configuration.}


}


\sframe{Empirical Data for Calibration}{


\begin{columns}
\column{0.6\textwidth}
\centering
\includegraphics[width=\textwidth]{figures/density_indics_morpho_discrquantiles}
\column{0.3\textwidth}
\centering
\includegraphics[width=\textwidth]{figures/density_cluster_pca_k5_morpho}\\
\includegraphics[width=\textwidth]{figures/density_cluster_map_k5_morpho}
\end{columns}

\justify

\footnotesize\textit{Computation of morphological indicators on population density data for Europe (shown here on France), morphological classification.}

}



\sframe{Model Calibration}{

\includegraphics[width=\textwidth]{figures/spatialsens_calibmorphogenesis}

\footnotesize\textit{Brute force calibration by exploring the parameter space. Reproduction of most existing configuration in the morphological sense (here in principal plan).}

}



\sframe{Model Targeted Exploration}{

\centering

\includegraphics[width=0.8\textwidth]{figures/spatialsens_pse.png}

\footnotesize\textit{Potentialities of targeted model explorations: here feasible space using Pattern Space Exploration algorithm \cite{10.1371/journal.pone.0138212}.}

}

\sframe{Other urban morphogenesis models}{

Benchmarking exponential mixtures, correlated percolation and gravity-based models \cite{raimbault2020comparison}

\medskip

\centering

\includegraphics[height=0.8\textheight]{figures/mesobenchmark_Fig3.png}


}

\sframe{Feasible morphological spaces}{

\includegraphics[height=0.9\textheight]{figures/mesobenchmark_Fig4.png}


}



\sframe{Network Generation model}{

Network generated conditionally to population; at fixed time steps :

\begin{enumerate}
	\item Add new nodes preferentially to new population and connect them
	\item \justify Variable heuristic for new links, among: nothing, random, gravity-based deterministic breakdown, gravity-based random breakdown (from \cite{schmitt2014modelisation}), cost-benefits (from \cite{louf2013emergence}), biological network generation (based on \cite{tero2010rules})
\end{enumerate}

\medskip

\centering

\frame{\includegraphics[width=0.32\textwidth]{figures/example_nwgrowth_tick0.png}}
\frame{\includegraphics[width=0.32\textwidth]{figures/example_nwgrowth_tick2.png}}
\frame{\includegraphics[width=0.32\textwidth]{figures/example_nwgrowth_tick10.png}}

}




\sframe{Biological network generation}{

Model studied by~\cite{tero2010rules} : exploration and reinforcement by a slime mould searching for ressources

\bigskip

\includegraphics[width=0.32\textwidth]{figures/slimemould_tick1}
\includegraphics[width=0.32\textwidth]{figures/slimemould_tick10}
\includegraphics[width=0.32\textwidth]{figures/slimemould_tick20}\\
\includegraphics[width=0.32\textwidth]{figures/slimemould_tick50}
\includegraphics[width=0.32\textwidth]{figures/slimemould_tick101}
\includegraphics[width=0.32\textwidth]{figures/slimemould_reseauFinal}\\

\medskip

\footnotesize
\textit{Application to the design of optimal bus routes}

}




\sframe{Biological Network generation}{

Adding new links with biological heuristic:

\begin{enumerate}
	\item Create network of potential new links, with existing network and randomly sampled diagonal lattice
	\item Iterate for $k$ increasing ($k\in \{ 1,2,4 \}$ in practice) :
	\begin{itemize}
		\item Using population distribution, iterate $k\cdot n_b$ times the slime mould model to compute new link capacities
		\item Delete links with capacity under $\theta_d$
		\item Keep the largest connected component
	\end{itemize}
	\item Planarize and simplify final network
\end{enumerate}

\medskip

\centering

\frame{\includegraphics[width=0.6\textwidth]{figures/7-1-1-fig-networkgrowth-bioexample.jpg}}

\footnotesize

\textit{Intermediate stage for biological network generation}

}



\sframe{Model : Specification}{

\includegraphics[width=\textwidth]{figures/coevol_mesocoevol}

}




\sframe{Generated Urban Shapes: Urban Form}{

\centering

\frame{\includegraphics[width=0.28\textwidth]{figures/coevol_example_synthsetup}}\hspace{0.1cm}
\frame{\includegraphics[width=0.28\textwidth]{figures/coevol_example_form-accessonly}}\hspace{0.1cm}
\frame{\includegraphics[width=0.28\textwidth]{figures/coevol_example_form-droadonly}}\\\vspace{0.1cm}
\frame{\includegraphics[width=0.28\textwidth]{figures/coevol_example_form-bwonly}}\hspace{0.1cm}
\frame{\includegraphics[width=0.28\textwidth]{figures/coevol_example_form-closenessonly}}\hspace{0.1cm}
\frame{\includegraphics[width=0.28\textwidth]{figures/coevol_example_form-poponly}}

\footnotesize\textit{In order: setup; accessibility driven; road distance driven; betweenness driven; closeness driven; population driven.}

}






\sframe{Generated Urban Shapes: Network}{


\centering

\frame{\includegraphics[width=0.28\textwidth]{figures/coevol_example_nw-connection}}\hspace{0.1cm}
\frame{\includegraphics[width=0.28\textwidth]{figures/coevol_example_nw-random}}\hspace{0.1cm}
\frame{\includegraphics[width=0.28\textwidth]{figures/coevol_example_nw-gravity}}\\\vspace{0.1cm}
\frame{\includegraphics[width=0.28\textwidth]{figures/coevol_example_nw-rndbrkdwn}}\hspace{0.1cm}
\frame{\includegraphics[width=0.28\textwidth]{figures/coevol_example_nw-cost}}\hspace{0.1cm}
\frame{\includegraphics[width=0.28\textwidth]{figures/coevol_example_nw-bio}}

\footnotesize\textit{In order: connection; random; deterministic breakdown; random breakdown; cost-driven; biological.}

}





\sframe{Synthetic systems of cities}{


\textit{At the macroscopic scale: synthetic systems of cities}


\begin{itemize}
	\item Evolutive urban theory: systems of cities follow general stylized facts \cite{pumain2018evolutionary}
	\item Rank-size law \cite{pumain2006evolutionary}
	\item Central place theory
\end{itemize}


}

\sframe{Cities and networks}{

% simpopnet generator

\textit{Synthetic system of cities and network for the SimpopNet model} \cite{raimbault2018unveiling}

\medskip

\begin{center}
\includegraphics[height=0.75\textheight]{figures/simpopnet_Fig1.png}
\end{center}


}

\sframe{Cities and networks}{

% ex from cities nw paper

\textit{Cities-network co-evolution model explored on synthetic systems of cities} \cite{raimbault2019modeling}

\medskip

\begin{center}
\includegraphics[height=0.47\textheight]{figures/macrocoevol_example_virtual_0_t0.png}\hspace{0.2cm}
\includegraphics[height=0.47\textheight]{figures/macrocoevol_example_virtual_0_tf.png}
\end{center}


}


\sframe{Co-evolution model}{

\includegraphics[width=\textwidth]{figures/macrocoevol_en.png}

\medskip

Raimbault, J. (2018). Modeling the co-evolution of cities and networks. Handbook of cities and networks, Rozenblat C., Niel Z., eds. (in press) arXiv:1804.09430.


}


\sframe{Urban evolution model}{


\begin{columns}
	\begin{column}{0.7\linewidth}
		\includegraphics[width=\linewidth]{figures/innovation_4.png}
	\end{column}
	\begin{column}{0.3\linewidth}
		\includegraphics[width=\linewidth]{figures/innovation_leg.png}
	\end{column}
\end{columns}

\medskip

Raimbault, J. (2020). A model of urban evolution based on innovation diffusion. Artificial Life Conference Proceedings 2020 NO. 32, 500-508.



}


\sframe{Other disciplines and spatial structures ?}{

% morphogensis : interdisc ?

% suggest: ecology (habitat / spatial niches ?) ; topography ; ?

\textit{Spatial synthetic data in other disciplines?}

\medskip

\begin{itemize}
	\item spatial network generative models
	\item other disciplines? ecology \cite{carraro2020generation}? geosciences \cite{mogheir2004characterizing}?
	\item interaction with data driven disciplines ? (planning, architecture, spatio-temporal datamining)
	\item genericity of some models ? (reaction-diffusion)
	\item synthetic data generation methods (synthetic populations)
	\item synthetic data at the core of applied statistics methodology (not much in spatial statistics ?)
\end{itemize}


}



\sframe{Spatial form and function: Morphogenesis}{

\vspace{-0.2cm}

\includegraphics[width=\textwidth,height=0.82\textheight]{figures/intro_examples}

\justify

\vspace{-0.4cm}

\tiny\textit{Sources (in order by column). Ants, Erosion, Game of Life: NetLogo Library ; Arbotron \cite{jun2005formation}; Industrial design \cite{Aage:2017aa}; Swarm chemistry \cite{sayama2007decentralized}}



}



\section{Perturbation of data}


\sframe{Real data perturbation}{

$\rightarrow$ \textit{How does noise in real data impacts the result ?}

% - WIP

\medskip

\begin{itemize}
	\item Impact of missing elements
	\item Impact of imprecise coordinates or topology
	\item Optimal matching between spatial datasets
\end{itemize}


\bigskip

$\rightarrow$ \textit{How does perturbation of real data allows to explore scenario}

\medskip

\textbf{Examples:}

\begin{itemize}
   \item simulating urban projects by modifying population of areas with a given spatial correlation structure %Forcity example
	\item simulating network disruptions or new transportation lines
\end{itemize}


}






\section{Sensitivity to spatial configuration}



\sframe{Method flowchart}{

\textit{General workflow to test the spatial sensitivity of simulation models}

\bigskip

\footnotesize

Raimbault, J., Cottineau, C., Le Texier, M., Le Nechet, F., \& Reuillon, R. (2019). Space Matters: Extending Sensitivity Analysis to Initial Spatial Conditions in Geosimulation Models. Journal of Artificial Societies and Social Simulation, 22(4).

\bigskip

\centering

\includegraphics[width=\textwidth]{figures/spatialsens_spacemattersworkflow.png}

}


\sframe{Quantification of spatial sensitivity}{

\textit{Relative distance of phase diagrams to compare global model behavior when meta-parameters change}

\medskip

\[
d_r\left(\mu_{\vec{\alpha}_1},\mu_{\vec{\alpha}_2}\right) = 2 \cdot \frac{d(\mu_{\vec{\alpha}_1},\mu_{\vec{\alpha}_2})^2}{Var\left[\mu_{\vec{\alpha}_1}\right] + Var\left[\mu_{\vec{\alpha}_2}\right]}
\]

}


\sframe{Application: Schelling model}{

\textit{Why could the Schelling model be sensitive to space ?}

\medskip

\cite{banos2012network} network effects in Schelling model

\medskip

\centering

\includegraphics[width=0.45\textwidth]{figures/spatialsens_schelling_ex0_t0.png}
\hspace{0.1cm}
\includegraphics[width=0.45\textwidth]{figures/spatialsens_schelling_ex0_t91.png}


}


\sframe{Sensitivity of the Schelling model}{

\textit{Influence of spatial generator parameters on model outputs}

\centering

\includegraphics[height=0.8\textheight]{figures/spatialsens_schellingreg.png}

}


\sframe{Application: Sugarscape model}{

\textit{A model of resource collection}

\medskip

\begin{itemize}
	\item agents collect a spatial resource
	\item the resource regrows at a certain rate only
\end{itemize}


\bigskip


\textit{Relative distances between phase diagrams}

\medskip

\includegraphics[width=0.49\textwidth]{figures/sugarscape_Fig4.png}
\includegraphics[width=0.49\textwidth]{figures/sugarscape_Fig5.png}

%  idea : compare with kind of saltelli approach ? and the dynamical system stuff ? -> would make an other paper


}



\section{Spatial indicators for model outputs}


\sframe{Spatial statistics}{

\textit{In the spatial approach, spatial model indicators are also important: what kind of spatial structure does the model produce ?}

\bigskip 
 
 \begin{itemize}
 	\item previous form indicators at different scales, applied on any spatialized variable or event: quantify level of aggregation, hierarchy, clustering
 	\item spatial statistics indicators and methods
 	\item more complicated approaches: fractals and multifractals, spatial datamining
 \end{itemize}

}





\sframe{Moran index}{

\textit{Spatial autocorrelation at a given range}

\medskip

Given spatial weights $w_{ij}$

\[
I = \frac{N}{\sum_{i,j} w_{ij}} \cdot \frac{\sum_{i,j}w_{ij} \cdot (X_i - \bar{X}) (X_j - \bar{X})}{\sum_i (X_i - \bar{X})^2}
\]

}


\sframe{Spatial autocorrelation regimes}{

% example from EnergyPrice ?

\textit{Extracting typical spatial scales of a system using Moran Index} \cite{bergeaud2020empirical}

\medskip

\includegraphics[width=\textwidth]{figures/moran_typicalscales.png}

}

\sframe{Ripley K function}{

\justify

\textit{Quantifying the level of clustering of spatial points or points on a network regarding a Poisson process null model} \cite{vandenbulcke2017location}


\medskip

\begin{center}
\includegraphics[width=0.72\textwidth]{figures/cybergeo_cyclistaccidents_img-2.png}
\end{center}


}


\sframe{Geographically Weighted Regression}{

\textit{GWR handles spatial non-stationarity with adaptative scale fitting} \cite{fotheringham2003geographically}

\medskip

\begin{center}
\includegraphics[width=\textwidth]{figures/gwr_energyprice.png}
\end{center}

\footnotesize

\nocite{bergeaud2020empirical}

Bergeaud, A., \& Raimbault, J. (2020). An empirical analysis of the spatial variability of fuel prices in the United States. Transportation Research Part A: Policy and Practice, 132, 131-143.


}


\section{Synthesis}

\sframe{Synthesis of available tools}{



\begin{columns}
	\begin{column}{0.45\linewidth}
	
	\bigskip
	\bigskip
	
	Micro grid spatial samplings
	
	\medskip
	
	Meso grid spatial samplings
	
	\medskip
	
	Macro spatial samplings
	
	\medskip

	Spatial network generation
	
	\medskip

	Real data import
	
	\medskip

	Real data perturbations
	
	\medskip

	Spatial statistics
	
	\medskip

	Hybrid methods
	
	\medskip

	Domain models (transportation, land-use)
		
	\end{column}
	
	
	
	\begin{column}{0.2\linewidth}
		
		\vspace{-0.3cm}
		
		\textbf{OpenMOLE}
		
		\medskip
		
		{\textcolor{green}\cmark}
		
		\medskip
		
		{\textcolor{green}\cmark}
		
		\medskip
		
		{\textcolor{red}\xmark}
			
			\medskip
		
		{\textcolor{red}\xmark}
		
		\medskip
		
		{\textcolor{red}\xmark}
		
		\medskip
		
		{\textcolor{red}\xmark}
		
		\medskip

		{\textcolor{blue}\cmark}
		
		\medskip
		
		{\textcolor{red}\xmark}
		
		\medskip
		
		{\textcolor{red}\xmark}
			
	\end{column}
	
	
	
	\begin{column}{0.25\linewidth}
		
		\vspace{-0.3cm}
		
		\textbf{spatialdata}
		
		\medskip
		
		{\textcolor{green}\cmark}
		
		\medskip
		
		{\textcolor{green}\cmark}
		
		\medskip
		
		{\textcolor{green}\cmark}
			
			\medskip
		
		{\textcolor{blue}\cmark}
		
		\medskip
		
		{\textcolor{green}\xmark}
		
		\medskip
		
		{\textcolor{red}\xmark}
		
		\medskip

		{\textcolor{blue}\cmark}
		
		\medskip
		
		{\textcolor{red}\xmark}
		
		\medskip
		
		{\textcolor{blue}\cmark}
	\end{column}
	
	
	
	\begin{column}{0.2\linewidth}
		
		\vspace{-0.3cm}
		
		\textbf{planned}
		
		\medskip
		
		{\textcolor{green}\cmark}
		
		\medskip
		
		{\textcolor{green}\cmark}
		
		\medskip
		
		{\textcolor{green}\cmark}
			
			\medskip
		
		{\textcolor{green}\cmark}
		
		\medskip
		
		{\textcolor{green}\cmark}
		
		\medskip
		
		{\textcolor{green}\cmark}
		
		\medskip
		
		{\textcolor{green}\cmark}
		
		\medskip
		
		{\textcolor{green}\cmark}
		
		\medskip
		
		{\textcolor{green}\cmark}
	\end{column}

	
	
	
\end{columns}



}



\section{Application on the Zombie model}


\sframe{Synthetic spaces for ZOMBIE}{

\textit{Try the GUI with synthetic spaces at }

\texttt{https://om.exmodelo.org/spatialsens/}

\medskip

\centering

\includegraphics[width=\textwidth]{figures/GUIZOMBIE_spatialsens.png}

}



\sframe{Application on the Zombie model}{


\justify

\textit{First application: fixed model parameters, sensitivity of time series to spatial configuration}

\bigskip
\bigskip

% spatialsens fixed params : work on derivative ! (clearer for the peak ?)

% show two different ways of implementing : either within the zombie model ; or with samplings (below is spatialdata library)

$\rightarrow$ explore the script \texttt{2{\_}spatialsens.oms}

}

\sframe{Sensitivity of zombified dynamics}{

\centering

\includegraphics[width=\textwidth]{figures/spatialsens_generators_zombified.png}

}


\sframe{Application on the Zombie model}{
%
\justify
%
\textit{Second application: variance-based comparison of phase diagrams for the cooperation submodel, between varying spatial configuration}
%
\bigskip
\bigskip
%
%% spatialsens fixed params : work on derivative ! (clearer for the peak ?)
%
% show two different ways of implementing : either within the zombie model ; or with samplings (below is spatialdata library)

$\rightarrow$ explore the script \texttt{3{\_}phasediag.oms}

\bigskip

\textbf{No results} for this experiment (problem of memory more than computation time) - method still need to be reformulated/implemented in a ``Saltelli-like'' manner


}


\sframe{Application on the Zombie model}{


\textit{Third application: calibration of the default parameters model, for each type of generator, to maximize peak size (time-localized dynamics) and minimize number of zombified}

\bigskip
\bigskip


$\rightarrow$ explore the script \texttt{4{\_}calibrate.oms}


}

\sframe{Calibration results}{

\footnotesize

\textit{Complementary restricted Pareto fronts between the different spatial generators}

\medskip

$\rightarrow$ urban planning policies yield different zombie-resilient compromises, and different ``planning paradigms'' are complementary

\bigskip

\includegraphics[width=0.48\textwidth]{figures/paretos_allsamples.png}
\includegraphics[width=0.48\textwidth]{figures/paretos_20samples.png}



}


\sframe{Going further: real world configurations for the zombie model}{

\textit{OpenStreetMap building sampler implemented in the spatialdata library}

\bigskip

$\rightarrow$ add the plugin \texttt{spatialdata{\_}2.13-0.4{\_}SNAPSHOT.jar} to your OpenMOLE instance

\medskip

$\rightarrow$ explore the script \texttt{5{\_}openstreetmap.oms}

\medskip

$\rightarrow$ integrate this sampling into the zombie model

\medskip

$\rightarrow$ simulate a zombie invasion for each district of a city of your choice


}



\sframe{Ideas for the challenge}{

% optimal placement of traps ? cf Paul's fiche but with systematic generators ?

\begin{itemize}
	\item test the robustness of your qualitative findings to the spatial configuration
	\item use spatial metrics to answer specific questions
	\item change the initial spatial distribution of agents (\textbf{\textit{need library/model tuning}})
	\item study the influence of spatial distribution of rescues or traps (\textbf{\textit{need library/model tuning if structured spatial configuration}})
	\item compute complicated spatial statistics on agents trajectories e.g. (\textbf{\textit{need library/model tuning}})
\end{itemize}


}


\sframe{Take home message}{

\justify

$\rightarrow$ \textbf{Space matters: } relevance of spatially-explicit models and spatial sensitivity analysis.

\bigskip

$\rightarrow$ \textbf{Synthetic data: } first experimental samplings included in OpenMOLE, soon more to come.

\bigskip

$\rightarrow$ \textbf{Disciplinary context: } strong contingency on included models and forms, please provide feedbacks, suggestions, needs, ideas from your viewpoint.

\medskip

\textit{Open issues at } \url{https://github.com/openmole/spatialdata/issues}

 

}


%\sframe{Submit to special session at CCS}{
%
%
%\includegraphics[width=\linewidth]{figures/ccs.png}
%
%\medskip
%
%\textit{Satellite session on methods and epistemology in modeling and simulation, at Conference on Complex Systems, 2nd October 2019}
%
%\smallskip
%
%\textbf{Submit your abstract before June 30th !}
%
%\url{https://iscpif.fr/ccs-satelllite-session-2019-new-methods/}
%
%\smallskip
%
%\textbf{Submission link:}
%
%\url{https://easychair.org/conferences/?conf=simexplo2019}
%
%}
%








\backupbegin


%%%%%%%%%%%%%%%%%%%%%
\begin{frame}[allowframebreaks]
\frametitle{References}
\bibliographystyle{apalike}
\bibliography{biblio}
\end{frame}
%%%%%%%%%%%%%%%%%%%%%%%%%%%%


\backupend





\end{document}


