\documentclass{../beamer_template/myBeamer}

\input{../beamer_template/header.tex}

\title[Case study model]{Case study: an epidemiological model}
%\subtitle{}
\author[]{Yu Si-te$^1$}
\date{June 24, 2019}
\institute{$^1$College of Public Health, Chongqing University\\
\includegraphics[width=0.3\textwidth]{figures/chongqing_medicaluniv}}

\begin{document}



\begin{frame}[plain]
	\titlepage
\end{frame}
\addtocounter{framenumber}{-1}

\AtBeginSection[]
{
	\frame{
		\tableofcontents[currentsection, hideallsubsections]
	}
	\addtocounter{framenumber}{-1}
}


% no need for outline
%\sframe{Outline}{
%\tableofcontents
%}

%Introducing a model 

%general purpose : spatial epidemio model

%scenarization

% - pedestrian dynamics



\sframe{Decision making in a chaotic world}{

 % contextualize Center for Zombie Research
 % knowledge forgotten about how to use simulation models ? -> you are the last hope of the world

\centering

\includegraphics[width=\linewidth,height=0.8\textheight]{figures/zombieOutbreak.png}

\justify

\textit{Simulation of the 2010 Zombie outbreak in the US} \cite{alemi2015you}

}


\sframe{History of Zombie epidemiology}{

\begin{itemize}
	\item 2007: first outbreak in Island, relatively contained through ad-hoc measures
	\item 2010: it becomes pandemic
	\item 2010-2015: no clear records of events
	\item 2015-2018: reorganization of institutions, the MOLE (Medical Overview of Ludicrous Experiments) center in Chongqing gathers observational from many local invasions across the world
	\item 2019: they released the first version of the model ZOMBIE (Zone of Optimal Management for Bacillus Infecting Everyone) and successfully applied
\end{itemize}

% 


}


\sframe{An operational model for local Zombie invasion}{

\justify



\begin{center}
%\includegraphics[width=\textwidth]{figures/zombieGUI.png}
\includegraphics[width=\textwidth]{figures/zombieland_gui.png}
\end{center}

\medskip

\footnotesize

\textit{Local scale agent-based model}

}

\sframe{Let's get your hands on it}{
  
  \begin{itemize}
  	\item A submodel is available at  \url{https://om.exmodelo.org/coop}. Try the GUI and changing parameters
  	\item Most of next courses will be based on that model (additional processes will be detailed when needed)
  \end{itemize}
  
}



\sframe{Overview of the model}{

\begin{itemize}
	\item Simulate agent-level collective movements at the scale of a district
	\item Include behavioral processes for human (panic, search for rescues, \ldots) and zombies (self-organization, spontaneous attacks, \ldots), which can be adapted to local settings
	\item Include realistic pedestrian dynamics and realistic spatial configuration, which can be applied to local configuration
\end{itemize}

\medskip

\justify

\textbf{Objective of the model: } optimal policies and behavioral prevention to minimize the impact of recurring invasions

\medskip

\textbf{Issue with model application: } model has many parameters and processes, model behavior is unknown, application may be strongly case-dependent

\medskip

$\rightarrow$ \textit{we need YOU to understand this model to save the world}


}

\sframe{Basic processes and parameters}{

\begin{itemize}
	\item Humans and Zombies walk/run randomly (smoothed random walk) in an open urban space (movement parameters: rotation angle, walk and run speed)
	\item Interactions: human flee from zombie, zombies run for food, fight when encounter
	\item Humans can be rescued and information on the existence of rescues propagates between humans
	\item Additional processes in a multi-modeling approach (army, vaccination, \ldots)
\end{itemize}

}


\sframe{Pedestrian simulation}{

  % a bit of literature on pedestrian models

Multiple approaches to pedestrian simulations:

\medskip

\begin{itemize}
	\item Social force models \cite{helbing1995social}
	\item Granular flows \cite{cristiani2011multiscale}
	\item Behavioral models \cite{antonini2006discrete}
	\item Cellular automatons \cite{burstedde2001simulation}
	\item Potential field \cite{jian2014perceived}
\end{itemize}

\medskip

\textit{The ZOMBIE model takes the last approach, relatively realistic in a panic setting}

}


\sframe{Agents state machines}{

\includegraphics[width=0.51\textwidth]{figures/humanStateMachine.png}\hspace{0.2cm}
\includegraphics[width=0.46\textwidth]{figures/zombieStateMachine.png}

% launch simu https://om.exmodelo.org/coop/

}

\sframe{Information and rescues}{
 
 \begin{itemize}
 	\item Some spots allows informed humans to be rescued and get out of the world
 	\item An initial ratio of humans \texttt{humanInformRatio} is informed of the existence of rescues
 	\item Informed humans which are not in a panicking state follow a specific potential field leading to rescues
 	\item A human can inform an other one at the same location with a probability \texttt{humanInformProbability}
 \end{itemize}
 
\medskip

\justify

With the additional parameter \texttt{humanFollowProbability} (probability for a human to begin running and follow when they encounter an other running human), the submodel with three parameters is aimed at studying cooperation between humans.
 
}


\sframe{A flexible and more general model}{

% one world on multi-modeling / deactivated parameters
% hidden parameters ?

}


\sframe{The model in practice}{
 % scala impl + GUI -> layus on implementation / platform

}




\sframe{The scala model}{

% main function zombieInvasion
% (transition for next course)

% say that plugin - as a jar


%\texttt{
import zombies._

    val rng = new scala.util.Random(replication)

    val result = zombieInvasion(
      humanFollowProbability = humanFollowProbability,
      humanInformedRatio =humanInformedRatio,
      humanInformProbability = humanInformProbability,
      zombies = 4,
      humans = 250,
      steps = 500,
      random = rng)

    val humansDynamic = result.humansDynamic(20)
    val zombiesDynamic = result.zombiesDynamic(20)
%}

}


\backupbegin




%%%%%%%%%%%%%%%%%%%%%
\begin{frame}[allowframebreaks]
\frametitle{References}
\bibliographystyle{apalike}
\bibliography{biblio}
\end{frame}
%%%%%%%%%%%%%%%%%%%%%%%%%%%%


\backupend





\end{document}


